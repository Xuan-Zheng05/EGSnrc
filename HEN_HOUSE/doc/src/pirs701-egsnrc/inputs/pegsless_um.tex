
%%%%%%%%%%%%%%%%%%%%%%%%%%%%%%%%%%%%%%%%%%%%%%%%%%%%%%%%%%%%%%%%%%%%%%%%%%%%%%%
%
%  EGSnrc manual: pegsless operation
%  Copyright (C) 2015 National Research Council Canada
%
%  This file is part of EGSnrc.
%
%  EGSnrc is free software: you can redistribute it and/or modify it under
%  the terms of the GNU Affero General Public License as published by the
%  Free Software Foundation, either version 3 of the License, or (at your
%  option) any later version.
%
%  EGSnrc is distributed in the hope that it will be useful, but WITHOUT ANY
%  WARRANTY; without even the implied warranty of MERCHANTABILITY or FITNESS
%  FOR A PARTICULAR PURPOSE.  See the GNU Affero General Public License for
%  more details.
%
%  You should have received a copy of the GNU Affero General Public License
%  along with EGSnrc. If not, see <http://www.gnu.org/licenses/>.
%
%%%%%%%%%%%%%%%%%%%%%%%%%%%%%%%%%%%%%%%%%%%%%%%%%%%%%%%%%%%%%%%%%%%%%%%%%%%%%%%
%
%  Author:          Blake Walters, 2013
%
%  Contributors:
%
%%%%%%%%%%%%%%%%%%%%%%%%%%%%%%%%%%%%%%%%%%%%%%%%%%%%%%%%%%%%%%%%%%%%%%%%%%%%%%%


% Replace commented line for the one with fixed date when commiting
% Beware: Using the macro below conflicts between CVS and latex!!!
% \lfoot[{\sffamily {\leftmark}}]{{\small Last edited $Date: 2013/09/26 17:59:27 $
\lfoot[{\sffamily {\leftmark}}]{{\small Last edited $Date: 2013/09/26 17:59:27 $
}}


\section{Pegsless Mode}
\label{pegsless}

As of 2013, EGSnrc user codes can be run in pegsless mode.  User codes run in this mode do not
require the use of interaction cross sections calculated a-priori using the PEGS4 code
({\em i.e.} no {\tt .pegs4dat} file).  Photon cross
sections have been generated on-the-fly since 2006.  Thus, migration to a fully pegsless implementation
is a logical next step.

\subsection{User Code Inputs for Pegsless Mode}
\label{pegslessinputsect}

To run a user code in pegsless mode, the user must include parameters for calculating cross-sections, including
specifications for all media used in the simulation, in their {\tt .egsinp} file between the
delimiters, {\tt :start media definition:} and {\tt :stop media definition:}.  This is best illustrated
with an example input:

\begin{verbatim}
:start media definition:

AE=0.521
UE=50.511
AP=0.01
UP=50.

material data file=/home/username/HEN_HOUSE/pegs4/data/material.dat

:start H2O521ICRU:
  elements = H, O
  number of atoms = 2,1
  rho = 1.0
  bremsstrahlung correction = KM
:stop H2O521ICRU:

:start AIR521ICRU:
  elements = C,N,O,AR
  mass fractions = 1.24000E-04, 7.55200E-01, 2.31800E-01, 1.28300E-02
  rho = 1.2048E-03
  bremsstrahlung correction = NRC
  gas pressure = 1.0
:stop AIR521ICRU:

:start PMMA521ICRU:
  bremsstrahlung correction = NRC
  density correction file = /home/uname/EGSnrc/HEN_HOUSE/pegs4/density_corrections/
compounds/polymethylmethacrylate__lucite___perspex___plexiglas_.density
:stop PMMA521ICRU:

:stop media definition:
\end{verbatim}
where:
\begin{description}
\item {\tt AE,UE,AP,UP } are the kinetic energy limits for calculating photon ({\tt AP,UP}) and electron ({\tt AE,UE}) cross sections in
\index{pegsless mode!AE}\index{pegsless mode!UE}\index{pegsless mode!AP}\index{pegsless mode!UP}
MeV.  These energy limits are also mentioned in the context of the EGSnrc system in Section~\ref{common_blocks}.
If {\tt AE} is
not specified it defaults to the highest value of {\tt ECUT} (electron transport cutoff energy--see Section~\ref{step_2}) specified
in the simulation.  If
{\tt AP} is unspecified, then it defaults to the highest value of {\tt PCUT} (photon transport cutoff energy--see Section~\ref{step_2})
specified.  {\tt UE} and {\tt UP} default to 50.511 MeV and 50.0 MeV, respectively.

\index{pegsless mode!material data file}
\item {\tt material data file} is the name (including full directory path) of a file containing specifications for the media used
in the simulation.  Provided with the EGSnrc distribution is the material data file {\tt \$HEN\_HOUSE/pegs4/data/material.dat} which
contains specifications necessary to reproduce all of the cross-section data in {\tt 521icru.pegs4dat} and {\tt 700icru.pegs4dat}
(provided that the appropriate values of {\tt AE, UE, AP and UP} are specified--see above).  Note that the format for
media specifications in the material data file is similar to that used for specifying media directly in the
{\tt .egsinp} file, described immediately below.  In the case of the material data file, however, there is some redundancy in the specification to allow the
user to see the composition and density of the media.

\index{pegsless mode!defining media in .egsinp file}
\item the {\tt :start MEDNAME:} and {\tt :stop MEDNAME:} delimiters are used to specify medium, {\tt MEDNAME}, directly in the
{\tt .egsinp} file.  This method of specifying a medium is used if {\tt MEDNAME} is not included in the material data file or
if you wish to override some or all of the specifications for {\tt MEDNAME} in the material data file.
If no material data file is input, then all media in the simulation
must be specified in this way.  The inputs between these delimiters are described below.  Variables
in square brackets are the analogous PEGS4 variables described in the PEGS4 Manual (Section~\ref{pegs4}) above.
\begin{description}
\index{pegsless mode!elements}
\item {\tt elements} specifies the elements comprising the medium.  Elements are specified using chemical symbols separated by
commas.  Case is unimportant.
\index{pegsless mode!no. of atoms}
\index{pegsless mode!mass fractions}
\item {\tt number of atoms} $[${\tt PZ}$]$ or {\tt mass fractions} $[${\tt RHOZ}$]$.  For each of the {\tt elements}, specify either the number of atoms in a molecule of the medium ({\it i.e.} stoichiometric coefficients), if the
medium is a compound, or the mass fractions of the elements in the medium,
if the medium is a mixture.
Values are separated by commas and are input in the same order as their corresponding elements.  In the example above,
the composition of {\tt H2O521ICRU} is
defined using the number of atoms of each element, while that of {\tt AIR521ICRU} is defined using the mass fraction of each element.  Note that this input
is omitted if the medium is an element.
\index{pegsless mode!rho}
\index{pegsless mode!bulk density}
\item {\tt rho} specifies the bulk density of the medium in g/cm$^3$.
\index{pegsless mode!stopping power type}
\index{pegsless mode!bremsstrahlung correction}
\index{pegsless mode!IAPRIM}
\item {\tt bremsstrahlung correction} $[${\tt IAPRIM}$]$ specifies the
correction to apply to calculated bremsstrahlung cross-sections.
Options are:
\begin{itemize}
\item {\tt KM} $[$IAPRIM=0$]$: Apply Koch and Motz\cite{KM59} empirical corrections.
\item {\tt NRC} $[$IAPRIM=1$]$: (the default) Apply NRC corrections based on NIST/ICRU\cite{Ro89a}.  These corrections are read from the file
{\tt \$HEN\_HOUSE/pegs4/aprime.data}.
\item {\tt None} $[$IAPRIM=2$]$: No corrections applied.
\end{itemize}
\index{pegsless mode!density correction file}
\index{pegsless mode!EPSTFL}
\item {\tt density correction file} $[${\tt EPSTFL}$]$ is the name of a file containing density effects which, when applied to calculated collision
stopping powers, results in agreement with collision stopping powers published in ICRU37\cite{ICRU37}.  In general, density correction files are specified including their full directory path and {\tt .density} file extension.  However,
the file can be specified by its prefix only if it
exists in, in order of search priority:
\begin{enumerate}
\item {\tt \$EGS\_HOME/pegs4/density\_corrections}
\item {\tt \$EGS\_HOME/pegs4/density\_corrections/elements}
\item {\tt \$EGS\_HOME/pegs4/density\_corrections/compounds}
\item {\tt \$EGS\_HOME/pegs4/density}
\item {\tt \$HEN\_HOUSE/pegs4/density\_corrections/elements}
\item {\tt \$HEN\_HOUSE/pegs4/density\_corrections/compounds}
\end{enumerate}
Note that the density correction files for many elements, compounds
and mixtures are supplied with
the distribution.  Density correction files have a header portion from which the composition and bulk density of the medium are read.  These values override
any user inputs for {\tt elements=}, {\tt number of atoms=} or {\tt mass fractions=}, and {\tt rho=}.  Thus, as in the case of
{\tt PMMA521ICRU} in the example above, it is possible to specify the composition
of a medium simply by specifying a density correction file.
\index{pegsless mode!gas pressure}
\index{pegsless mode!GASP}
\item {\tt gas pressure} $[${\tt GASP}$]$ is the pressure of the medium in atm
if the medium is a gas.  This input is only relevant (and necessary
for a gas) if
a density correction file is not used, in which case {\tt gas pressure} is used to modify the calculated density effect parameters.
{\tt gas pressure} defaults to 0 ({\em i.e.} the medium is not a gas).
\end{description}
\end{description}

Inputs specifying media are case insensitive with the exception of the medium name ({\it e.g.} {\tt H2O521ICRU} is
different than {\tt h2o521ICRU}).

For more details on the inputs for specifying a medium, please refer to the PEGS4 Manual above (Section~\ref{pegs4}).

\index{pegsless mode!running code}
\subsection {Running User Codes in Pegsless Mode}

EGSnrc user codes that read an input file can be run interactively in pegless mode using the command line input:
\begin{verbatim}
user_code -i inputfile
\end{verbatim}
where {\tt inputfile} is the name of the input file (with no
{\tt .egsinp} extension).

Pegsless batch runs use the command line syntax:
\begin{verbatim}
exb user_code inputfile pegsless [short|medium|long] [batch=batch_system] [p=N]
\end{verbatim}
This is identical to the syntax for a batch run with pegs data but
with the word ``{\tt pegsless}'' in place of the name of the pegs data file.

\index{.mederr file}
When running in pegsless mode, EGSnrc outputs a file, {\tt inputfile.mederr}, which, for each medium used
in the simulation, indicates where each specifying parameter has been read ({\it i.e.} from a material data
file or directly from the {\tt .egsinp} file).  The file also includes warnings when {\tt AE}, {\tt UE},
{\tt AP} or {\tt UP} have not been specified and have been set to their default values and when a material
data file has not been specified.  For parallel runs in pegsless mode (parameter {\tt p} $>$ 1 in the batch
command syntax above), the {\tt .mederr} file is only output by the first job.

The actual values of the media specifications (including defaults) used to calculate cross-sections are
output in the listing file, {\tt inputfile.egslst}, and to the screen for interactive runs or in the
log file, {\tt inputfile.egslog}, for batch runs.

\subsection{Implementation of Pegsless Code}
\index{pegsless mode!implementation}

To run an EGSnrc user code in pegsless mode, the following files, all located in
{\tt \$HEN\_HOUSE/src}, must be included at compile time:

\begin{description}
\item {\tt pegs4\_macros.mortran}:
\index{pegs4\_macros.mortran}
\begin{itemize}
\item Contains PEGS4 common block variables used to calculate electron cross sections by subroutines in {\tt pegs4\_routines.mortran}
(see below).  In many cases these are modified versions of the common blocks used by {\tt pegs4.mortran} (the
original PEGS4 code).
\index{\$GET\_PEGSLESS\_XSECTIONS}
\item Contains the macro {\tt \$GET\_PEGSLESS\_XSECTIONS}.  This is coding that is inserted at compile time into
the subroutine {\tt HATCH} in {\tt egsnrc.mortran} and is executed when the code is run in pegsless mode.  First, this
block of coding calls the subroutine {\tt get\_media\_inputs} (see below) to read the pegsless inputs.  The coding then
loops over all media in the simulation and, for each medium:
\begin{enumerate}
\item copies media parameters stored in EGSnrc common block variables by\\
 {\tt get\_media\_inputs} into the corresponding
PEGS4 common block variables used by the cross section calculation subroutines
\item calls the
necessary subroutines in {\tt pegs4\_routines.mortran} for calculating electron cross sections
\item copies cross section data back out from the PEGS4 common block variables into the corresponding EGSnrc common
block variables for use in the simulation.
\end{enumerate}
Finally, {\tt show\_media\_parameters} is called (see below) to output the media specifications read in to standard
output (the {\tt .egslog} file in the case of a batch run).
\item Contains the macro {\tt \$DECLARE-PEGS4-COMMON-BLOCKS}, called at the beginning of {\tt HATCH} to declare the
PEGS4 common block variables that {\tt HATCH} needs to have access to, so that they can be copied from/to their
EGSnrc common block analogs.  This macro also declares some local and external types used by {\tt HATCH} in
pegsless mode.
\item Contains the macro {\tt \$INIT-PEGS4-VARIABLES} used to define some constants in the PEGS4 common blocks.
\item Must be included after {\tt egsnrc.macros} and before the user source code.
\end{itemize}
\index{pegs4\_routines\_mortran}
\item{\tt pegs4\_routines.mortran}:  This contains the subroutines necessary to calculate electron cross sections.  In most
cases, these subroutines have been copied directly from {\tt pegs4.mortran} with some modifications to variable names, suppressed
output, and the handling of density correction files.  Subroutines that appear directly in the {\tt \$GET\_PEGSLESS\_XSECTIONS}
coding block (see above) are:
\begin{itemize}
\index{MIX}
\item {\tt MIX}: This subroutine calculates multiple scattering parameters used to calculate electron cross sections.
\index{SPINIT}
\item {\tt SPINIT}: This subroutine determines the density correction to apply to electron stopping powers.  It takes as
an argument the name of the density correction file for the medium (if specified).  It then determines the appropriate
density corrections as outlined in Subsection~\ref{pegslessinputsect} above.
\index{PWLF1}
\item {\tt PWLF1}: This subroutine performs a piece-wise linear fit of functions describing bremsstrahlung, Moeller, Bhaba
and positron annihilation cross sections over the electron kinetic energy interval defined by $[${\tt AE, UE}$]$.  The subroutine
determines the number of sub-intervals required for the fits and returns a discrete point for each interaction type for
each sub-interval.
\end{itemize}
\index{get\_media\_inputs.mortran}
\item {\tt get\_media\_inputs.mortran}:
\begin{itemize}
\index{get\_media\_inputs}
\item Contains the {\tt get\_media\_inputs(ounit)} subroutine, which interprets
the pegsless inputs between the delimiters, {\tt :start media definition:} and {\tt :stop media definition:}
in the {\tt .egsinp} file.  The subroutine is also responsible for assigning default values for input parameters
that are out of range, omitted, or left blank.  Input values are then passed to the appropriate common block
variables for use in calculating/modifying electron cross sections for the media in the simulation. The parameter,
{\tt ounit}, specifies the unit number to which media specifications, once read, will
be echoed.  Usually this is unit number 6 (standard output).  If {\tt ounit}$\leq$0, then the media specs are not
echoed.  The subroutine also has an entry point, {\tt show\_media\_parameters(ounit)}, which can be called from
the user code to echo the media specifications to any unit at any point after they have been read in.  Finally,
this subroutine opens up and writes to the {\tt .mederr} file, containing information about the source of
media specifications used ({\it i.e.} the material data file or the {\tt .egsinp} file).
\index{GET\_INPUT\_PLUS}\index{GET\_INPUT}
\item Contains the subroutine {\tt GET\_INPUT\_PLUS}, which is a modified version of {\tt GET\_INPUT} (in the
file {\tt get\_inputs.mortran}).  Modifications include the ability to read a unit other than standard input
(the {\tt .egsinp} file) and the ability to have different starting and ending delimiters within which to search
for inputs.  {\tt GET\_INPUT\_PLUS} is used by {\tt get\_media\_inputs} to read and interpret the material data.
file.
\end{itemize}
\end{description}

\index{pegless mode!single precision}
Apart from ease of bookkeeping, the main reason for separating the PEGS4 common block variables used to
calculate cross sections from their EGSnrc common block analogs is to allow calculation of cross sections
with the same precision ({\tt REAL*4}) as {\tt pegs4.mortran}.  This conveniently allows comparisons between
results in pegsless mode and those using PEGS4 data files (created with identical media specifications and over
the same energy range).  Note, however, that there will still be differences within uncertainty between the two calculations.
This is because the reading of cross section data from fixed-format PEGS4 data files necessarily results in a
loss of precision compared to passing values internally from the PEGS4 common block variables to the EGSnrc
variables.

\index{pegless mode!double precision}
To change electron cross section calculations to double precision, go into \\
{\tt \$HEN\_HOUSE/src/pegs4\_macros.mortran}
and change the macro, {\tt \$REAL4} from {\tt real*4} to {\tt real*8} and recompile your user code.

\subsection{Changes to EGSnrc Source Codes}
\index{pegsless mode!changes to EGSnrc codes}

Implementation of pegsless mode necessitated the following changes to EGSnrc source coding:
\begin{description}
\item In {\tt \$HEN\_HOUSE/src/egsnrc.macros}:
\index{egsnrc.macros}
\begin{itemize}
\index{\$GET-PEGSLESS-XSECTIONS}\index{\$INIT-PEGS4-VARIABLES}\index{\$DECLARE-PEGS4-COMMON-BLOCKS}
\item Definition of empty macros {\tt \$GET-PEGSLESS-XSECTIONS}, {\tt \$INIT-PEGS4-VARIABLES} and {\tt \$DECLARE-PEGS4-COMMON-BLOCKS}
which are used in place of their versions defined in {\tt \$HEN\_HOUSE/src/pegs4\_macros.mortran} when user codes are compiled
without pegsless implementation.
\index{pegsless mode!is\_pegsless variable}
\index{is\_pegsless}
\item Definition of a new logical variable, {\tt is\_pegsless}, in the {\tt EGS-IO} common block.  This variable is set to {\tt .true.}
for pegsless runs (see below).
\end{itemize}
\index{egsnrc.mortran}
\item In {\tt \$HEN\_HOUSE/src/egsnrc.mortran}:
\begin{itemize}
\item Introduction of {\tt \$INIT-PEGS4-VARIABLES} and {\tt \$DECLARE-PEGS4-COMMON-BLOCKS} macros at the beginning of
subroutine {\tt HATCH}.
\item Within {\tt HATCH}, the introduction of a new conditional statement where
the {\tt .pegsdat} file is read if {\tt is\_pegsless=.false.} and the {\tt \$GET-PEGSLESS-XSECTIONS} macro
is executed if {\tt is\_pegsless=.true.}
\end{itemize}
\index{egs\_utilities.mortran}
\item In {\tt \$HEN\_HOUSE/src/egs\_utilities.mortran}
\begin{itemize}
\index{egs\_check\_arguments}
\item Subroutine {\tt egs\_check\_arguments} sets {\tt is\_pegsless=.true.} if the
``{\tt -p pegsfilename}'' argument is not supplied on the command line when running a user code.
\index{egs\_init1}
\item Subroutine {\tt egs\_init1} now only opens the PEGS4 data file if {\tt is\_pegsless=.false.}
\end{itemize}
\end{description}
