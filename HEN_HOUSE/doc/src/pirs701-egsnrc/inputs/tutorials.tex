
%%%%%%%%%%%%%%%%%%%%%%%%%%%%%%%%%%%%%%%%%%%%%%%%%%%%%%%%%%%%%%%%%%%%%%%%%%%%%%%
%
%  EGSnrc manual: tutorials
%  Copyright (C) 2015 National Research Council Canada
%
%  This file is part of EGSnrc.
%
%  EGSnrc is free software: you can redistribute it and/or modify it under
%  the terms of the GNU Affero General Public License as published by the
%  Free Software Foundation, either version 3 of the License, or (at your
%  option) any later version.
%
%  EGSnrc is distributed in the hope that it will be useful, but WITHOUT ANY
%  WARRANTY; without even the implied warranty of MERCHANTABILITY or FITNESS
%  FOR A PARTICULAR PURPOSE.  See the GNU Affero General Public License for
%  more details.
%
%  You should have received a copy of the GNU Affero General Public License
%  along with EGSnrc. If not, see <http://www.gnu.org/licenses/>.
%
%%%%%%%%%%%%%%%%%%%%%%%%%%%%%%%%%%%%%%%%%%%%%%%%%%%%%%%%%%%%%%%%%%%%%%%%%%%%%%%
%
%  Author:          Iwan Kawrakow, 2003
%
%  Contributors:    Blake Walters
%                   Frederic Tessier
%                   Ernesto Mainegra-Hing
%
%%%%%%%%%%%%%%%%%%%%%%%%%%%%%%%%%%%%%%%%%%%%%%%%%%%%%%%%%%%%%%%%%%%%%%%%%%%%%%%


\vspace*{-3mm}
% Replace line with fixed date with the one below when commiting
% Beware: Using the macro below conflicts between CVS and latex!!!
% \lfoot[{\sffamily {\leftmark}}]{{\small Last edited $Date: 2013/01/04 14:38:47 $
\lfoot[{\sffamily {\leftmark}}]{{\small Last edited 2011/03/09 19:35:20
}}

\label{tutorials}
\index{tutorial programs}
\index{examples!user codes}

EGSnrc is a powerful system which has been used in order to produce some
very complex Monte Carlo simulations.  In spite of this complexity the
user's interface with the system is, in principle, very simple.  In the
following series of tutorial programs we use various aspects of these
user interfaces in what we refer to as EGSnrc User Codes.  In these
User Codes we will introduce some of the basic scoring techniques and,
at the same time, will demonstrate the power of the Mortran3 language.
Formal documentation in the form of EGSnrc and PEGS4 User Manuals can be
found in sections~\ref{ERM} and ~\ref{pegs4}, respectively.  An EGSnrc
User Guide to Mortran3 can be found in section~\ref{UGM3} and an overview
of the system considerations is given in section~\ref{sys_consid}. With
the introduction of the EGSnrcMP environment, the user has a more
flexible interface at the system level, but that is described fully in
Report PIRS-877\cite{Ka03}.


These tutorials are written on the assumption that the reader is
generally familiar with the contents of the EGSnrc Reference Manual
(section~\ref{ERM} of this manual, page~\pageref{ERM}), although a
complete understanding is not required.  In fact, the purpose
of these tutorials is to make these manuals more understandable.
Although the programs presented here are very simple in construction,
it should become clear that with various extentions to them, generally
of a bookkeeping nature, a wide range of important problems can be
studied.  In the following sections, sometimes only partial listings of
User Codes are presented.  The complete source for each (and the
corresponding Fortran 77 code) is found on the EGSnrc Distribution.

Note that default parameters in EGSnrc may change from time to time, in which
case there could be some minor differences when you run the calculations. Also,
use of the MP system requires a Step 0: {\tt call egs\_init;} and a Step 9:
{\tt call egs\_finish;} which are in the distributed versions but not updated
here. \index{\$REAL}

% To see the expected output, see
% {\tt \$HEN\_HOUSE/test\_distribution\_outputs/}.

\index{tutor\_data.pegs4dat}
To run the tutorials, a special PEGS4 data set called {\tt
tutor\_data.pegs4dat} is on the distribution at {\tt
\$HEN\_HOUSE/pegs4/data}.  To execute the tutorials issue the
command:
\begin{verbatim}
      ex tutor1 "" tutor_data    or      tutor1 -p tutor_data
\end{verbatim}
where the {\tt ""} signifies that no input file is required (except for
{\tt tutor6, tutor7}). Note that with the EGSnrcMP environment
there are two other alternatives described
in PIRS-877\cite{Ka03} (\viz, the right hand form above or use the {\tt
egs\_gui}).  To compile the tutor codes, copy any one to your
own user-code area if it is not already there, eg:
\vspace*{-3mm}
\begin{verbatim}
                cd $EGS_HOME/tutor1
                cp $HEN_HOUSE/tutor1/tutor1.mortran .
                mf tutor1
\end{verbatim}
\vspace*{-3mm}
As alternatives to the {\tt mf} command, when using the EGSnrcMP
environment, one may use the {\tt egs\_gui} or simply issue the command
{\tt make} from {\tt \$EGS\_HOME/tutor1}(see PIRS-877\cite{Ka03}).
The meaning of the commands {\tt mf} and {\tt ex} are discussed in
section~\ref{sys_consid} (page~\pageref{sys_consid}).

\vspace*{-5mm}
\subsection{tutor1.mortran: 20 MeV e$^-$ through 1~mm of Ta}
\index{tutor1.mortran}
\vspace*{-3mm}
The geometry of all the tutorials is the same,  namely, a
semi-infinite slab of material is placed in a vacuum and a pencil beam
of photons or electrons is incident normal to the surface.  The slab
is in the X-Y plane and the particles are incident at the origin
travelling along the Z-axis.  In the first problem, a beam of 20 MeV
electrons is incident on a 1 mm thick plate of tantalum.  In order to
use EGSnrc to answer the question ``What comes out the far side of the
plate?'', we have created the user code {\tt tutor1.mortran} shown below.

\begin{latexonly}
%\input{tutor_codes_for_um/tut1_nc.mortran}
\input{inputs/tutor_codes_for_um/tut1MP_nc.mortran}
\end{latexonly}
\begin{htmlonly}
\clearpage
\input{inputs/tutor_codes_for_um/tut1MP_nc.mortran_html}
\clearpage
\end{htmlonly}

Needless to say, the above User Code listing is
somewhat difficult to read, and therefore confusing, in
spite of the fact that it is complete.  The following is
a heavily commented version of the same code, where the structure and
readability of the Mortran3 language clearly demonstrates itself.

\newpage
\begin{latexonly}
%\input{tutor_codes_for_um/tutor1.mortran}
\input{inputs/tutor_codes_for_um/tutor1_MP.mortran}
\clearpage
%\input{tutor_codes_for_um/tutor1_output_new}
\input{inputs/tutor_codes_for_um/tutor1_output_MP}
\clearpage
\end{latexonly}

\begin{htmlonly}
\clearpage
%\input{tutor_codes_for_um/tutor1.mortran_html}
\input{inputs/tutor_codes_for_um/tutor1_MP.mortran_html}
\clearpage
%\input{tutor_codes_for_um/tutor1_output_new_html}
\input{inputs/tutor_codes_for_um/tutor1_output_MP_html}
\clearpage
\end{htmlonly}

By keeping track of many of these histories, we could answer a lot of
questions about what comes out the far side of the plate, but it
should be recognised that these are all bookkeeping extensions to the
problem---the physics itself is already accomplished with EGSnrc and the
relatively small amount of User Code listed above.  The scoring
routine for this problem is the simplest possible; namely, it outputs
on the terminal some of the parameters of the various particles
leaving the plate.

In addition, this User Code includes examples of the following items
that are discussed further in the EGSnrc User Manual (section~\ref{ERM}).
\begin{itemize}
\item Defining simple macro replacements for templates
(e.g. the character string {\tt \$MXMED}
is replaced by 1 everywhere in the EGSnrc system).
%
\item The use of {\tt COMIN} statements (which is an EGSnrc macro to allow
easy insertion of {\tt COMMONS}).

\item The use of {\tt \$REAL, \$IMPLICIT-NONE, \$INTEGER}.
\index{\$REAL} \index{\$INTEGER} \index{\$IMPLICIT-NONE}
%
\item The technique required in order to define the array {\tt MEDIA}

\item The use of {\tt OUTPUT} statements (which are an easy way to output
things to Fortran Unit 6).
%
\item The definition of calling parameters for the SHOWER routine.
%
\item A very simple AUSGAB routine.

\item The replacement of {\tt \$CALL-HOWNEAR(\#)} with the recommended
subroutine call.

\item Simple HOWFAR and HOWNEAR subroutines.

\item Relational and logical operators such as
{\tt >= (.GE.), \& (.AND.), $\sim$= (.NE.)} and\\ {\tt  | (.OR.)}.  The only
warning is not to mix modes since this will generate errors (e.g.,
don't use {\tt IF(A=B.AND.C=D))}.
\index{relational operators}
\index{Mortran3!relational operators}
\end{itemize}


\subsection{tutor2.mortran: energy transmitted, reflected, deposited}
\index{tutor2.mortran}

In this example we use the same geometry as above, but we want to score
the fraction of the incident energy that is reflected from, transmitted
through, and deposited in the plate using the default parameter settings.
The coding is essentially the same as in {\tt tutor1.mortran} except that
{\tt COMMON/SCORE/} and a new array {\tt ESCORE} are defined at Step 1.
The latter is initialised to zero (Step 2) and subsequently printed
out on the line printer (Step~8).  The AUSGAB routine is considerably
different as shown below.

\begin{latexonly}
\input{inputs/tutor_codes_for_um/tutor2_ausgab.mortran}
\end{latexonly}
\begin{htmlonly}
\clearpage
\input{inputs/tutor_codes_for_um/tutor2_ausgab.mortran_html}
\clearpage
\end{htmlonly}

AUSGAB is still very simple since all we need to do is to keep track
of the energy deposited in the three regions.  The variable {\tt EDEP}
(available through {\tt COMMON/EPCONT/}) contains the energy deposited
during a particular step for a variety of different {\tt IARG}-situations,
as described in the comments in fig~\ref{fig_tutor2_ausgab} and further
elaborated upon in section~\ref{ausgab}(page~\pageref{ausgab}).  In this
example, but not always, we can sum {\tt EDEP} for any value of {\tt
IARG} up to 4.  Figure~\ref{fig_tutor2_output} shows the output from
{\tt tutor2.mortran}.

\newpage

\begin{latexonly}
%\input{tutor_codes_for_um/tutor2_output_new}
\input{inputs/tutor_codes_for_um/tutor2_output_MP}
\end{latexonly}
\clearpage
\begin{htmlonly}
\clearpage
\input{inputs/tutor_codes_for_um/tutor2_output_new_html}
\clearpage
\end{htmlonly}

\subsection{tutor3.mortran: NaI response function}
\index{tutor3.mortran}
\index{response function}
\index{AUSGAB!response function}


The geometry in  this  example is similar to the previous two but the
problem is very different.  Here we investigate the energy response
function for a 2.54 cm thick slab of NaI when a 5 MeV beam of photons
is incident on it.  In  this case the final scoring and binning is
done at the end of each history (\ie , after all the descendants
from each initial photon have been tracked completely).
Figure~\ref{fig_tutor3}
shows the changes required (at Steps 7 and 8) and the new AUSGAB
routine. Figure~\ref{fig_tutor3_output} shows the output from this code.
For a detailed discussion of the use of EGS to calculate response functions
see Rogers (1984)\cite{Ro82}.  The user code DOSRZnrc calculates response
functions in any cylindrical geometry\cite{Ro00}.



\begin{latexonly}
\input{inputs/tutor_codes_for_um/tutor3.mortran}
\newpage
\input{inputs/tutor_codes_for_um/tutor3_output_new}
\clearpage
\end{latexonly}

\begin{htmlonly}
\clearpage
\input{inputs/tutor_codes_for_um/tutor3.mortran_html}
\clearpage
\input{inputs/tutor_codes_for_um/tutor3_output_new_html}
\clearpage
\end{htmlonly}

\subsection{tutor4.mortran: use of SUBROUTINE WATCH}
\label{tutor4}
\index{tutor4.mortran}
\index{SUBROUTINE WATCH}
\index{WATCH}
\index{IWATCH}


The {\tt tutor4.mortran} user code is functionally identical to {\tt
tutor2.mortran}, \ie~ it scores the total amount of energy reflected,
deposited and transmitted when a 20 MeV beam of electrons is incident
on a 1~mm slab of Ta.  However it has the added ability to use the
auxiliary {\tt SUBROUTINE WATCH} which is part of the file {\tt
\$HEN\_HOUSE/nrcaux.mortran}.  As set up, {\tt tutor4.mortran} outputs
to the terminal, detailed information about each particle interaction
which occurs during the simulation.

Implementing the use of {\tt WATCH} consists of two mandatory calls and one
optional call, which has been included in {\tt tutor4.mortran}.   These
extra calls are shown in fig~\ref{fig_tutor4}. Figure~\ref{fig_watch} shows
the header of {\tt SUBROUTINE WATCH} which is part of {\tt nrcaux.mortran}.
This listing describes how to use {\tt SUBROUTINE WATCH}.

Figure~\ref{fig_tutor4_output} shows portions of the output from {\tt
tutor4}.  Note that the full output requires 132 columns and this figure
has been edited slightly so the font size is still more or less visible!
It is well worth studying this output carefully and being sure that you
understand what is happening in the shower.


\index{EGS\_Windows}
\index{.egsgph}
In {\tt tutor4} the value of {\tt IWATCH} has been set to 1 and hence the output
lists every time an interaction occurs or a particle is discarded for some
reason.  If one changes the value of {\tt IWATCH} to 2 the output includes
information on every single electron step and hence explodes dramatically
in quantity.  Finally, if one has installed the code {\tt EGS\_Windows} for
doing a 3-D graphics display of EGS simulations, then {\tt IWATCH = 4} will
write a files (to unit 13) which is the input for this code.  This
site also presents a series of examples of the use of {\tt EGS\_Windows}.
It should be noted that starting with Version 4 of the {\tt EGS\_Windows}
system, it works on any X-windows based Unix/Linux system\cite{TR99a}.
Figure~\ref{fig_tutor4_environment} presents an example of the environment
file needed with {\tt tutor4} if it is to write a data file for display by
the {\tt EGS\_Windows} system.  In this case the file is named {\tt
tutor4.egsgph}.
\index{examples!.environment file}
\index{environment file}
\index{.environment file}

All users are strongly advised to build {\tt SUBROUTINE WATCH} into any
user code that they write since it has proven absolutely invaluable for
debugging programs.  When things don't work, being able to track a few
histories inevitably helps to sort out the problem.

Although not apparent in this example, WATCH also displays when
bremsstrahlung photons are split, when relaxation particles are created
(fluorescent photons, Auger electrons, Coster-Kronig electrons) or when
particles are discarded by internal Russian Roulette.

{\bfseries One major warning: when using WATCH you must only use it for a
few histories at a time due to the large quantity of output!}



\begin{latexonly}
\input{inputs/tutor_codes_for_um/tutor4_new.mortran}
\input{inputs/tutor_codes_for_um/watch_header.mortran}
\input{inputs/tutor_codes_for_um/tutor4_output_54_new}
\input{inputs/tutor_codes_for_um/tutor4.io}
\clearpage
\end{latexonly}

\begin{htmlonly}
\clearpage
\input{inputs/tutor_codes_for_um/tutor4_new.mortran_html}
\clearpage
\input{inputs/tutor_codes_for_um/watch_header_new.mortran_html}
\clearpage
\input{inputs/tutor_codes_for_um/tutor4_output_54_new_html}
\clearpage
\input{inputs/tutor_codes_for_um/tutor4.io_html}
\clearpage
\end{htmlonly}




\subsection{tutor5.mortran: using LATCH and Rayleigh scattering}
\index{tutor5.mortran}
%
\index{LATCH}
This tutorial program is an example that includes Rayleigh scattering
and which makes use of the variable {\tt LATCH} (contained in
{\tt COMMON/STACK/}).
{\tt LATCH} can be set for any particle on the ``stack" of particles being
transported, and it is passed on to all its progeny.  This provides a
simple procedure for keeping track of the history of a particle.  In
this case we make use of {\tt LATCH} to keep track of how often photons
from an incident 50 keV beam are Compton or Rayleigh scattered while
passing through a 0.5 cm slab of water.

\index{IAUSFL}
The program also demonstrates the use of the {\tt IAUSFL} array of
flags (in {\tt COMMON /EPCONT/}).  By setting the appropriate flags,
the user can cause the EGSnrc system to call the AUSGAB subroutine in
any combination of 28 well specified situations (see section~\ref{ausgab},
page~\pageref{ausgab}).
By default, EGSnrc calls AUSGAB only for 5 out of the
possible 28 situations.  Here, by setting {\tt IAUSFL(18)} and
{\tt IAUSFL(24)} from 0 (default) to 1 in the main program, we cause EGSnrc
to call AUSGAB with {\tt IARG=17} and {\tt IARG=23} (\ie, just before a
Compton or a Rayleigh scattering event, respectively).  We make use of
these calls to set some flags associated with each photon rather than
for scoring any variables.  The AUSGAB routine also makes use of a few
simple local macros in order to make the logic of the coding more
transparent.  Perhaps the greatest strength of Mortran3 is this
ability to create readable and hence more accurate coding.
A complete listing of {\tt tutor5.mortran}, except for the HOWFAR/HOWNEAR
routines
which are similar to the other examples, is given below.
\index{AUSGAB} \index{IARG}

Note that the logic in {\tt tutor5.mortran} for EGSnrc
is the same as it was for EGS4.  This is only possible because in water
there are no fluorescent photons above the 10 keV PCUT value in the
problem.  If the material were changed to a higher Z material, e.g. lead,
one would need to change the logic since the present logic allows
fluorescent photons created via relaxation after a Compton interaction to
be scored as a Compton scattered photon.

\begin{latexonly}
%\input{tutor_codes_for_um/tutor5.mortran}
\input{inputs/tutor_codes_for_um/tutor5_MP.mortran}
\input{inputs/tutor_codes_for_um/tutor5_output_54_new}
%note under MP results identical
\clearpage
\end{latexonly}

\begin{htmlonly}
\clearpage
\input{inputs/tutor_codes_for_um/tutor5.mortran_html}
\clearpage
%following has been updated
\input{inputs/tutor_codes_for_um/tutor5_output_50_html}
\clearpage
\end{htmlonly}


\typeout{************start tutor6.mortran}
\subsection{tutor6.mortran: modifying the transport options}
\label{tutor6}
\index{tutor6.mortran}
\index{transport options}
\index{changing transport options}

\index{EGS4}
\index{PRESTA}
As seen from the previous tutorial programs, EGSnrc can be run in the
default mode without making any selections regarding transport algorithms
or about which features to include. In these cases it it uses default
algorithms which apply in the vast majority of cases.  However, there are
some options which are wasteful to model in many simulations, but
critical for others.  For example, taking into account binding effects in
Compton scattering can be important for low-energy calculations but has no
effect for most cases above 1 MeV. The same is true for Rayleigh (coherent)
scattering of photons. In addition, there have been many changes made to the
electron transport algorithm in EGS4/PRESTA and EGSnrc.  To our knowledge
the default simulations with EGSnrc provide accurate electron transport but
to allow us and others to understand the differences between EGSnrc and
EGS4/PRESTA, we have left hooks in the EGSnrc code which allow various
old transport options to still be used (see
section~\ref{mimic},page~\pageref{mimic}). So, for example, EGSnrc can be run
with the PRESTA-I electron transport algorithm turned on instead of the new
PRESTA-II transport algorithm.  Also one can chose to vary the boundary
crossing algorithm used in EGSnrc and mimic that used in EGS4/PRESTA.  This
can be useful for investigations about the impact of the changes on a given
user code.  More importantly, it allows some of the time consuming features
of the simulation in the default EGSnrc to be ``turned off'' if it turns
out that they do not affect the users problem.

The {\tt tutor6.mortran} code does a simple calculation, very similar to
those in the previous tutorial codes, but asks the user for a series of
options to use.  In fact, this little code gives an example of how to
change every EGSnrc transport option that we can think of.  The purpose is
to show how easy it is. Basically you must assign a value to some flag or
variable and
ensure that the appropriate {\tt COMIN} is defined in the subroutine or
main routine where the inputs are read.

The variables which are set are summarised in Table~\ref{tab_tutor6} and
figure~\ref{fig_tutor6_parts} shows one part of code which inputs two of
these variables (please see the full version of the code on the
distribution at {\tt \$HEN\_HOUSE/tutor6/tutor6.mortran}).

{\tt tutor6.mortran} also exhibits two other interesting features. It
initialises two variables, viz {\tt IREJECT} and {\tt ESAVE},  which
allow EGSnrc's internal range rejection to be utilised by setting the
arrays {\tt i\_do\_rr} and {e\_max\_rr}.  By setting {\tt IREJECT} to 1,
EGSnrc terminates the history of any electron below the energy ESAVE
if that particle is unable to reach the closest geometric boundary.
See section~\ref{range_rejection} (page~\pageref{range_rejection})
for more information.  {\tt tutor6.mortran} also does a statistical
analysis of its results using a new procedure suggested by Francesc
Salvat\cite{SB00} which doesn't use the common ``batch technique''
but scores the statistics on an event by event basis in an efficient
manner. This procedure is shown in figure~\ref{fig_tutor6_stats}.

Note that if {\tt tutor6.mortran} is run with bremsstrahlung splitting
turned on it will be seen that the total energy is not exactly 1.0. This
is because of the lack of exact energy conservation as discussed in
section~\ref{brem_splitting} (page~\pageref{brem_splitting}). As expected,
as the number of histories increases, the energy conservation gets closer
to unity.

\begin{table}
\index{ibr\_nist}
\index{bca\_algorithm}
\begin{center}
\caption{The transport control variables set in {\tt tutor6.mortran} and
the {\tt COMINs} they are contained in. These represent all user controllable
transport variables. See the source code of
{\tt tutor6.mortran} and
sections~\ref{common_blocks} and \ref{step_2}
for more detailed descriptions of the meaning of each.\vspace{3mm} }
\label{tab_tutor6}
\begin{tabular}{ll}
\hline
&\\
Variable & {\tt COMIN} \\
&\\
\hline
&\\
ECUT,PCUT   &  BOUNDS\\
&\\
IBRDST      & BREMPR\\
IPRDST       & BREMPR\\
ibr\_nist     & BREMPR \\
&\\
IBCMP        & COMPTON-DATA~~~~~~~~~\\
&\\
IEDGFL      & EDGE\\
IPHTER      & EDGE\\
&\\
ESTEPE      & ET-CONTROL \\
XIMAX			&ET-CONTROL \\
SKINDEPTH\_FOR\_BCA	&ET-CONTROL \\
TRANSPORT\_ALGORITHM~~~~~~~~~~~~~~	&ET-CONTROL \\
BCA\_ALGORITHM		&ET-CONTROL \\
EXACT\_BCA		&ET-CONTROL \\
SPIN\_EFFECTS		&ET-CONTROL \\
&\\
IRAYLR   		&MISC\\
&\\
luxury\_level,iseed      & none \\
&\\
\hline
\end{tabular}
\end{center}
\end{table}


\begin{table}
\begin{center}
\caption{The intrinsic variance reduction parameters set in {\tt
tutor6.mortran}. They are included in EGSnrc via {\tt
COMIN/EGS-VARIANCE-REDUCTION}.}
\begin{htmlonly}
\rule[-0.0mm]{15cm}{0.1mm}
\end{htmlonly}
\begin{tabular}{c}
\hline
\\
~~~~~~~~~~~~~~~~~~~~~~~~~~~~~~~~~Variable~~~~~~~~~~~~~~~~~~~~~~~~~~~~~~~~~  \\
\\
\hline
\\
nbr\_split \\
i\_play\_RR	\\
prob\_RR	\\
i\_do\_rr	\\
e\_max\_rr	\\
\hline
\end{tabular}
\end{center}
\end{table}
\clearpage

%\input{tutor_codes_for_um/tutor6.mortran}
\begin{latexonly}
\input{inputs/tutor_codes_for_um/eg_tutor6.mortran}
\input{inputs/tutor_codes_for_um/ausgab_tutor6.mortran}
\clearpage
\end{latexonly}

\begin{htmlonly}
\clearpage
\input{inputs/tutor_codes_for_um/eg_tutor6.mortran_html}
\clearpage
\input{inputs/tutor_codes_for_um/ausgab_tutor6.mortran_html}
\clearpage
\end{htmlonly}

\index{examples!use of input file}
\index{tutor6.mortran}
\index{transport options}
\index{changing transport options}

{\tt tutor6.mortran} is designed to be run interactively, prompting for
responses from the terminal.  This can become tedious and to shorten the
process the scripts used to run EGSnrc allow the user to specify an input
file which is read as if it were input from the terminal.
Figure~\ref{fig_tutor6_input} shows an example of such an input file to be
used with {\tt tutor6.mortran}.  The only trouble is that the prompts
become rather garbled since the code was not written to handle this case.
Figure~\ref{fig_tutor6_output_50} shows the terminal session when
{\tt tutor6.mortran} is run interactively using the same inputs as in the
sample input file.

\begin{latexonly}
\index{.egsinp}
\input{inputs/tutor_codes_for_um/5mev_e_1mm_Ta.egsinp}
\clearpage
%\input{tutor_codes_for_um/tutor6_output_54_new}
\input{inputs/tutor_codes_for_um/tutor6_output_54_MP}
\clearpage
\end{latexonly}

\begin{htmlonly}
\clearpage
\index{.egsinp}
\input{inputs/tutor_codes_for_um/5mev_e_1mm_Ta_html}
\clearpage
\input{inputs/tutor_codes_for_um/tutor6_output_54_new_html}
\clearpage
\end{htmlonly}

\newpage
%
\par\noindent
%
\subsection{tutor7.mortran: using SUBROUTINE GET\_INPUTS}
\index{GET\_INPUTS}
\index{SUBROUTINE GET\_INPUTS}
\index{tutor7.mortran}

{\tt tutor7.mortran} is like the same as {\tt tutor6.mortran} but it
uses {\tt SUBROUTINE GET\_INPUTS} to input the necessary inputs.  This is a
powerful package which allows for a very descriptive input format which
makes the creation of input files much simpler.  This example is given here
to encourage people to learn about {\tt SUBROUTINE GET\_INPUTS} which is
described in detail in another manual\cite{Ro00}.


%\mylistfile{TUTOR7A.TEX}
%
\subsection{Sophisticated User Codes}
\index{NRC User Codes}
\index{FLURZnrc}
\index{DOSRZnrc}
The EGSnrc distribution includes a variety of NRC written general purpose user codes such as
DOSRZnrc, FLURZnrc etc.  These are described in a separate manual ``NRC User Codes for EGSnrc''
\cite{Ro00}. These provide many working examples of how to do more sophisticated operations such as
handle cylindrical geometries, properly score fluence, use complex LATCH algorithms, handle
statistical analysis etc.  The BEAM system is for
modelling linear accelerators and doing dose calculations in a CT-based patient
phantom\cite{Ro95,Ro98a,Ma01b}.
